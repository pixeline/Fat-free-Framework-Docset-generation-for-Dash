\href{http://fatfree.sf.net/}{\tt }

{\bfseries A powerful yet easy-\/to-\/use P\+HP micro-\/framework designed to help you build dynamic and robust \hyperlink{class_web}{Web} applications -\/ fast!}

\href{https://flattr.com/submit/auto?user_id=phpfatfree&url=https://github.com/bcosca/fatfree}{\tt }

Condensed in a single $\sim$65\+KB file, \hyperlink{class_f3}{F3} (as we fondly call it) gives you solid foundation, a mature code base, and a no-\/nonsense approach to writing \hyperlink{class_web}{Web} applications. Under the hood is an easy-\/to-\/use \hyperlink{class_web}{Web} development tool kit, a high-\/performance U\+RL routing and cache engine, built-\/in code highlighting, and support for multilingual applications. It\textquotesingle{}s lightweight, easy-\/to-\/use, and fast. Most of all, it doesn\textquotesingle{}t get in your way.

Whether you\textquotesingle{}re a novice or an expert P\+HP programmer, \hyperlink{class_f3}{F3} will get you up and running in no time. No unnecessary and painstaking installation procedures. No complex configuration required. No convoluted directory structures. There\textquotesingle{}s no better time to start developing \hyperlink{class_web}{Web} applications the easy way than right now!

\hyperlink{class_f3}{F3} supports both S\+QL and No\+S\+QL databases off-\/the-\/shelf\+: My\+S\+QL, S\+Q\+Lite, M\+S\+S\+Q\+L/\+Sybase, Postgre\+S\+QL, D\+B2, and Mongo\+DB. It also comes with powerful object-\/relational mappers for data abstraction and modeling that are just as lightweight as the framework. No configuration needed.

That\textquotesingle{}s not all. \hyperlink{class_f3}{F3} is packaged with other optional plug-\/ins that extend its capabilities\+:-\/


\begin{DoxyItemize}
\item Fast and clean template engine,
\item Unit testing toolkit,
\item Database-\/managed sessions with automatic C\+S\+RF protection,
\item Markdown-\/to-\/\+H\+T\+ML converter,
\item Atom/\+R\+SS feed reader,
\item \hyperlink{class_image}{Image} processor,
\item Geodata handler,
\item Google static maps,
\item On-\/the-\/fly Javascript/\+C\+SS compressor,
\item Open\+ID (consumer),
\item Custom logger,
\item Basket/\+Shopping cart,
\item Pingback server/consumer,
\item Unicode-\/aware string functions,
\item \hyperlink{class_s_m_t_p}{S\+M\+TP} over S\+S\+L/\+T\+LS,
\item Tools for communicating with other servers,
\item And more in a tiny supercharged package!
\end{DoxyItemize}

Unlike other frameworks, \hyperlink{class_f3}{F3} aims to be usable -\/ not usual.

\href{https://flattr.com/submit/auto?user_id=phpfatfree&url=https://github.com/bcosca/fatfree}{\tt }

The philosophy behind the framework and its approach to software architecture is towards minimalism in structural components, avoiding application complexity and striking a balance between code elegance, application performance and programmer productivity.

\href{https://www.paypal.com/cgi-bin/webscr?cmd=_s-xclick&hosted_button_id=MJSQL8N5LPDAY}{\tt }



\subsection*{Table of Contents}


\begin{DoxyItemize}
\item \href{#getting-started}{\tt Getting Started}
\item \href{#routing-engine}{\tt Routing Engine}
\item \href{#framework-variables}{\tt Framework Variables}
\item \href{#views-and-templates}{\tt Views and Templates}
\item \href{#databases}{\tt Databases}
\item \href{#plug-ins}{\tt Plug-\/\+Ins}
\item \href{#optimization}{\tt Optimization}
\item \href{#unit-testing}{\tt Unit Testing}
\item \href{#quick-reference}{\tt Quick Reference}
\item \href{#support-and-licensing}{\tt Support and Licensing}
\end{DoxyItemize}

\href{https://twitter.com/phpfatfree}{\tt }

\subsubsection*{Version 3.\+6 Is Finally Released!}

The latest official release welcomes the summer with a bang and marks the final milestone in this version of the Fat-\/\+Free Framework. Packed with exciting new features and outstanding documentation that consumed significant time and effort to develop and refine, version 3.\+6 is now available for download. This edition is packed with a bunch of new usability and security features.

\hyperlink{class_f3}{F3} has a stable enterprise-\/class architecture. Unbeatable performance, user-\/friendly features and a lightweight footprint. What more can you ask for?

It is highly recommended that experienced users develop new applications with this version to take advantage of the latest code base and its significant improvements.

\subsection*{Introducing Fat\+Free\+Framework.\+com}

{\bfseries Detailed A\+PI documentation with lots of code examples and a graphic guide can now be found at \href{http://fatfreeframework.com/}{\tt http\+://fatfreeframework.\+com/}.}

Of course this handy online reference is powered by F3! It showcases the framework\textquotesingle{}s capability and performance. Check it out now.

\subsection*{Getting Started}

\begin{quote}
{\itshape A designer knows he has achieved perfection not when there is nothing left to add, but when there is nothing left to take away. -- Antoine de Saint-\/\+Exupéry} \end{quote}


Fat-\/\+Free Framework makes it easy to build entire \hyperlink{class_web}{Web} sites in a jiffy. With the same power and brevity as modern Javascript toolkits and libraries, \hyperlink{class_f3}{F3} helps you write better-\/looking and more reliable P\+HP programs. One glance at your P\+HP source code and anyone will find it easy to understand, how much you can accomplish in so few lines of code, and how powerful the results are.

\hyperlink{class_f3}{F3} is one of the best documented frameworks around. Learning it costs next to nothing. No strict set of difficult-\/to-\/navigate directory structures and obtrusive programming steps. No truck load of configuration options just to display {\ttfamily \textquotesingle{}Hello, World\textquotesingle{}} in your browser. Fat-\/\+Free gives you a lot of freedom -\/ and style -\/ to get more work done with ease and in less time.

\hyperlink{class_f3}{F3}\textquotesingle{}s declarative approach to programming makes it easy for novices and experts alike to understand P\+HP code. If you\textquotesingle{}re familiar with the programming language Ruby, you\textquotesingle{}ll notice the resemblance between Fat-\/\+Free and Sinatra micro-\/framework because they both employ a simple Domain-\/\+Specific Language for Re\+S\+Tful \hyperlink{class_web}{Web} services. But unlike Sinatra and its P\+HP incarnations (Fitzgerald, Limonade, Glue -\/ to name a few), Fat-\/\+Free goes beyond just handling routes and requests. Views can be in any form, such as plain text, H\+T\+ML, X\+ML or an e-\/mail message. The framework comes with a fast and easy-\/to-\/use template engine. \hyperlink{class_f3}{F3} also works seamlessly with other template engines, including Twig, Smarty, and P\+HP itself. Models communicate with \hyperlink{class_f3}{F3}\textquotesingle{}s data mappers and the S\+QL helper for more complex interactions with various database engines. Other plug-\/ins extend the base functionality even more. It\textquotesingle{}s a total \hyperlink{class_web}{Web} development framework -\/ with a lot of muscle!

\subsubsection*{Enough Said -\/ See For Yourself}

Unzip the contents of the distribution package anywhere in your hard drive. By default, the framework file and optional plug-\/ins are located in the {\ttfamily lib/} path. Organize your directory structures any way you want. You may move the default folders to a path that\textquotesingle{}s not Web-\/accessible for better security. Delete the plug-\/ins that you don\textquotesingle{}t need. You can always restore them later and \hyperlink{class_f3}{F3} will detect their presence automatically.

{\bfseries Important\+:} If your application uses A\+PC, Memcached, Win\+Cache, X\+Cache, or a filesystem cache, clear all cache entries first before overwriting an older version of the framework with a new one.

Make sure you\textquotesingle{}re running the right version of P\+HP. \hyperlink{class_f3}{F3} does not support versions earlier than P\+HP 5.\+3. You\textquotesingle{}ll be getting syntax errors (false positives) all over the place because new language constructs and closures/anonymous functions are not supported by outdated P\+HP versions. To find out, open your console ({\ttfamily bash} shell on G\+N\+U/\+Linux, or {\ttfamily cmd.\+exe} on Windows)\+:-\/


\begin{DoxyCode}
/path/to/php -v
\end{DoxyCode}


P\+HP will let you know which particular version you\textquotesingle{}re running and you should get something that looks similar to this\+:-\/


\begin{DoxyCode}
PHP 5.3.15 (cli) (built: Jul 20 2012 00:20:38)
Copyright (c) 1997-2012 The PHP Group
Zend Engine v2.3.0, Copyright (c) 1998-2012 Zend Technologies
\end{DoxyCode}


Upgrade if necessary and come back here if you\textquotesingle{}ve made the jump to P\+HP 5.\+3 or a later release. If you need a P\+HP 5.\+3+ hosting service provider, try one of these services\+:


\begin{DoxyItemize}
\item \href{http://www.a2hosting.com/2461-15-1-72.html}{\tt A2 Hosting}
\item \href{http://www.dreamhost.com/r.cgi?665472}{\tt Dream\+Host}
\item \href{http://hostek.com/aff.php?aff=364&plat=L}{\tt Hostek}
\item \href{http://www.siteground.com/index.htm?referrerid=155694}{\tt Site\+Ground}
\end{DoxyItemize}

\subsubsection*{Hello, World\+: The Less-\/\+Than-\/\+A-\/\+Minute Fat-\/\+Free Recipe}

Time to start writing our first application\+:-\/


\begin{DoxyCode}
$f3 = require('path/to/base.php');
$f3->route('GET /',
    function() \{
        echo 'Hello, world!';
    \}
);
$f3->run();
\end{DoxyCode}


Prepend {\ttfamily \hyperlink{base_8php_source}{base.\+php}} on the first line with the appropriate path. Save the above code fragment as {\ttfamily \hyperlink{index_8php_source}{index.\+php}} in your \hyperlink{class_web}{Web} root folder. We\textquotesingle{}ve written our first \hyperlink{class_web}{Web} page.

The first command tells the P\+HP interpreter that you want the framework\textquotesingle{}s functions and features available to your application. The {\ttfamily \$f3-\/$>$route()} method informs Fat-\/\+Free that a \hyperlink{class_web}{Web} page is available at the relative U\+RL indicated by the slash ({\ttfamily /}). Anyone visiting your site located at {\ttfamily \href{http://www.example.com/}{\tt http\+://www.\+example.\+com/}} will see the {\ttfamily \textquotesingle{}Hello, world!\textquotesingle{}} message because the U\+RL {\ttfamily /} is equivalent to the root page. To create a route that branches out from the root page, like {\ttfamily \href{http://www.example.com/inside/}{\tt http\+://www.\+example.\+com/inside/}}, you can define another route with a simple {\ttfamily G\+ET /inside} string.

The route described above tells the framework to render the page only when it receives a U\+RL request using the H\+T\+TP {\ttfamily G\+ET} method. More complex \hyperlink{class_web}{Web} sites containing forms use other H\+T\+TP methods like {\ttfamily P\+O\+ST}, and you can also implement that as part of a {\ttfamily \$f3-\/$>$route()} specification.

If the framework sees an incoming request for your \hyperlink{class_web}{Web} page located at the root U\+RL {\ttfamily /}, it will automatically route the request to the callback function, which contains the code necessary to process the request and render the appropriate H\+T\+ML stuff. In this example, we just send the string {\ttfamily \textquotesingle{}Hello, world!\textquotesingle{}} to the user\textquotesingle{}s \hyperlink{class_web}{Web} browser.

So we\textquotesingle{}ve established our first route. But that won\textquotesingle{}t do much, except to let \hyperlink{class_f3}{F3} know that there\textquotesingle{}s a process that will handle it and there\textquotesingle{}s some text to display on the user\textquotesingle{}s \hyperlink{class_web}{Web} browser. If you have a lot more pages on your site, you need to set up different routes for each group. For now, let\textquotesingle{}s keep it simple. To instruct the framework to start waiting for requests, we issue the {\ttfamily \$f3-\/$>$run()} command.

{\bfseries Can\textquotesingle{}t Get the Example Running?} If you\textquotesingle{}re having trouble getting this simple program to run on your server, you may have to tweak your \hyperlink{class_web}{Web} server settings a bit. Take a look at the sample Apache configuration in the following section (along with the Nginx and Lighttpd equivalents).

{\bfseries Still having trouble?} Make sure the `\$f3 = require(\textquotesingle{}path/to/base.\+php\textquotesingle{});{\ttfamily assignment comes before any output in your script.}\hyperlink{base_8php_source}{base.\+php}` modifies the H\+T\+TP headers, so any character that is output to the browser before this assignment will cause errors.

\subsection*{Routing Engine}

\subsubsection*{Overview}

Our first example wasn\textquotesingle{}t too hard to swallow, was it? If you like a little more flavor in your Fat-\/\+Free soup, insert another route before the {\ttfamily \$f3-\/$>$run()} command\+:-\/


\begin{DoxyCode}
$f3->route('GET /about',
    function() \{
        echo 'Donations go to a local charity... us!';
    \}
);
\end{DoxyCode}


You don\textquotesingle{}t want to clutter the global namespace with function names? Fat-\/\+Free recognizes different ways of mapping route handlers to O\+OP classes and methods\+:-\/


\begin{DoxyCode}
class WebPage \{
    function display() \{
        echo 'I cannot object to an object';
    \}
\}

$f3->route('GET /about','WebPage->display');
\end{DoxyCode}


H\+T\+TP requests can also be routed to static class methods\+:-\/


\begin{DoxyCode}
$f3->route('GET /login','Auth::login');
\end{DoxyCode}


Passed arguments are always provided as the second parameter\+:

``` php \$f3-\/$>$route(\textquotesingle{}G\+ET /hello/ 